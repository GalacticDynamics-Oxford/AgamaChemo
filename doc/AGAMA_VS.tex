\input pmacro \reportformat
\centerline{\titlefont Compiling AGAMA with MS Visual Studio}

\bigskip\noindent
Here I explain how to get Vasiliev's  AGAMA package (Action-based  Galaxy
Modelling Framework) running with Microsoft's freely-distributed `Visual Studio'
compiler. This is unfortunately essential if you want to run AGAMA under
Windows because it seems impossible to compile either Python or the Gnu
Scientific Library with the Gnu compilers, and a single compiler must compile all
components of a code.

Instructions for installing GSL can be found at

\noindent https://solarianprogrammer.com/2020/01/26/getting-started-gsl-gnu-scientific-library-windows-macos-linux/

In a Linux/Gnu environment, AGAMA is embodied in shared library {\tt agama.so}. Under
VS it is contained in two files {\tt agama.lib} and {\tt agama.dll}. The former is
what the linker uses, while the latter is required at runtime, so has to lie
on your {\tt path}. I have the following file {\tt vscl.bat} on my {\tt path}
so I can compile and link a program {\tt testit.cpp} that uses AGAMA by
typing {\tt vscl testit}
\begintt
@echo off
cl.exe /c /openmp /I\u\c\agama\agama -EHsc %1.cpp
@if not %ERRORLEVEL% == 0 (goto end)
link %1.obj agama.lib /LIBPATH:\u\c\agama\agama\x64\release /NOIMPLIB
del %1.obj
del %1.exp
:end
\endtt
What follows {\tt /I} in the second line is the path to
my AGAMA source code \& what follows {\tt LIBPATH:} in the fourth line is the path to {\tt
agama.lib}. That's where the VS IDE put it: I used the VS IDE to compile AGAMA
rather than hacking the file {\tt makefile.local} distributed with AGAMA and using {\tt
make}. The IDE handles access to the Gnu Scientific Library and Python if the latter were
installed using the {\tt vcpkg} utility.
The last {\tt del} command is necessary because VS insists on producing an
`export' library file {\tt.exp} even when told to
make an executable. 

\bigskip\centerline{\bf Installing AGAMA}

\noindent  Untar everything into your source directory and start the Visual Studio IDE and open {\tt agama.sln} (sln stands for `solution') within the
source directory. Then provided you've installed the Gnu Scientific Library within VS, you
should be able to `build agama' from within the `build' tab of the VS IDL. VS is a bossy
compiler so there will be zillions of warnings. It will even declare an error on correct
code, such one using {\tt fopen} rather than {\tt fopen\_s}.  It's easier to identify any
errors if you use the drop-down menu at the top centre of the error list to change to
`Build only'. 

You may need to set some compiler options. At the bottom of the Project tab, click on
Properties. Then click on C/C++ and by clicking on open MP support get Yes (/openmp)

After building AGAMA with Visual Studio Community 2019, Version 16.9.2 
I tested the procedure for installing AGAMA by transferring the code to an older machine
running Version 16.4.2. I encountered two problems: (i) min was said not to be a member of
sdt -- I fixed the problem by adding {\tt \#include <algorithm>} to {\tt math\_base.h}; (ii)
the linker complained about twoPhase template instantiation. This problem was fixed by
clicking
Project$\rightarrow$Properties$\rightarrow$C/C++$\rightarrow$Command line and typing in {\tt /Zc:twoPhase-} at the bottom.
This click sequence could also be used to establish what the compiler and linker flags
should be if you prefer to hack {\tt makefile.local} rather than using the IDE.
 

Among the many outputs of an error-free compilation  will be {\tt agama.dll}, which you need to copy to a folder
that lies on your path, and {\tt agama.lib}. The outputs will be in {\tt x64$\backslash$release}
or {\tt x64$\backslash$debug} if you've set VS to debugging (top left centre of IDE). Here I'm assuming that you
have a 64-bit machine and that you will be running the code from the `x64 Native Tools
Command Prompt' that should be in your Visual Studio folder on your start menu. If you
have a 32-bit machine, change the IDE from `x64' to `x86' at the top. 

The following lines in a makefile
\begintt
ASRC = \u\c\agama\agama
ALIB  = \u\c\agama\agama\x64\release\agama.lib
ADLL  = \u\c\agama\agama\x64\release\agama.dll
testit.exe: testit.cpp  $(ALIB) $(ADLL) other.obj
	cl /openmp /I$(ASRC) -EHsc testit.cpp /link /out:testit.exe NOIMPLIB \ 
	other.obj \libs\press.lib $(ALIB)
\endtt
will update testit.exe when any of {\tt testit.cpp}, {\tt other.obj} or AGAMA is changed. {\tt testit} will
be linked to AGAMA, other.obj and the static library
{\tt$\backslash$libs$\backslash$press.lib}. The words before {\tt/link} are instructions
to the compiler, while those after are instructions to the linker.

For some reason VS doesn't put any symbols from the dll it is constructing into the
{\tt.lib} file unless explicitly instructed to do so. So any symbol you want to reference
from code that's not in the AGAMA library has to have its name decorated by {\tt
\_\_declspec(dllexport)}. I've done this for most routines in the AGAMA library by adding

{\tt\#define EXP \_\_declspec(dllexport)} 

\noindent in header files and then placing {\tt EXP} at
appropriate places. In the case of a class, one writes

{\tt class EXP some\_class$\{$..}

\noindent which will cause VS to put into the {\tt.lib} file objects and methods defined
in the class definition. When a method's code is given outside the class definition, its name needs
another {\tt EXP}:

{\tt EXP double some\_class::some\_method(int i,..)$\{$..}

\noindent even though the method was of course declared when the class was defined.
Failure to decorate the name when the code is given generates a `redefinition with
different linkage' error. If the compiler can figure out the need for decoration, why the
hell can't it add it without bothering me?

\noindent The syntax for a {\tt struct} mirrors that of a {\tt class}

{\tt struct EXP some\_struct$\{$..}

\noindent Functions unconnected with a class are handled like methods define outside a
class

{\tt EXP double do\_something(double x,double y)$\{$..}

\noindent The older of my versions of Visual Studio complained when the names of symbols in internal
namespaces were decorated. Since these symbols should only be referenced from within the
library, this should not be a problem. Anyway don't add {\tt EXP} within internal
namespaces.

If you reference a symbol in the library that's not had its name thus decorated,
it will be missing from the {\tt.lib} file and the linker will report `unresolved symbols'. 
I haven't put {\tt EXP}s in the {\tt raga} files or {\tt py\_wrapper} because I haven't
used these bits of code. I have run all the {\tt test\_xx} programs in the distributed {\tt
tests} folder and checked that the results are comparable to those obtained (on a
different machine) with Gnu.

\bsk
\line{James Binney\hfil May 2021}

\bye 